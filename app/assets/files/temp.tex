\PassOptionsToPackage{unicode=true}{hyperref} % options for packages loaded elsewhere
\PassOptionsToPackage{hyphens}{url}
%
\documentclass[
]{article}
\usepackage{lmodern}
\usepackage{amssymb,amsmath}
\usepackage{ifxetex,ifluatex}
\ifnum 0\ifxetex 1\fi\ifluatex 1\fi=0 % if pdftex
  \usepackage[T1]{fontenc}
  \usepackage[utf8]{inputenc}
  \usepackage{textcomp} % provides euro and other symbols
\else % if luatex or xelatex
  \usepackage{unicode-math}
  \defaultfontfeatures{Scale=MatchLowercase}
  \defaultfontfeatures[\rmfamily]{Ligatures=TeX,Scale=1}
\fi
% use upquote if available, for straight quotes in verbatim environments
\IfFileExists{upquote.sty}{\usepackage{upquote}}{}
\IfFileExists{microtype.sty}{% use microtype if available
  \usepackage[]{microtype}
  \UseMicrotypeSet[protrusion]{basicmath} % disable protrusion for tt fonts
}{}
\makeatletter
\@ifundefined{KOMAClassName}{% if non-KOMA class
  \IfFileExists{parskip.sty}{%
    \usepackage{parskip}
  }{% else
    \setlength{\parindent}{0pt}
    \setlength{\parskip}{6pt plus 2pt minus 1pt}}
}{% if KOMA class
  \KOMAoptions{parskip=half}}
\makeatother
\usepackage{xcolor}
\IfFileExists{xurl.sty}{\usepackage{xurl}}{} % add URL line breaks if available
\IfFileExists{bookmark.sty}{\usepackage{bookmark}}{\usepackage{hyperref}}
\hypersetup{
  pdfborder={0 0 0},
  breaklinks=true}
\urlstyle{same}  % don't use monospace font for urls
\setlength{\emergencystretch}{3em}  % prevent overfull lines
\providecommand{\tightlist}{%
  \setlength{\itemsep}{0pt}\setlength{\parskip}{0pt}}
\setcounter{secnumdepth}{-2}
% Redefines (sub)paragraphs to behave more like sections
\ifx\paragraph\undefined\else
  \let\oldparagraph\paragraph
  \renewcommand{\paragraph}[1]{\oldparagraph{#1}\mbox{}}
\fi
\ifx\subparagraph\undefined\else
  \let\oldsubparagraph\subparagraph
  \renewcommand{\subparagraph}[1]{\oldsubparagraph{#1}\mbox{}}
\fi

% set default figure placement to htbp
\makeatletter
\def\fps@figure{htbp}
\makeatother


\date{}

\begin{document}

If Chauntecleer's first tale falls apart at the end, the structure of
his next is more solid, however his intentions with this story become
strikingly clear. In this tale, pulled from the same source as the last,
again there are two men are on a voyage, this time preparing to embark
on a ship. One of the men dreams the night before the voyage of a
nondescript messenger who tells him curtly, ``\ldots{} `If thou
tomorrowe wend, / Thow shalt be dreynt; my tale is at an ende'\,''
(4271-72). This dreamer believes the prophecy, while his companion is
both the skeptic and the victim, sailing out the next day, refusing to
believe the dreamer's convictions, and consequently drowning at sea.

The succinct language of the dream messenger is emblematic of the short,
purposeful style of this tale. 35 lines shorter than the first tale, the
amount of detail in this tale is sparse, as Chauntecleer only tells what
is directly relevant to the story. On first reading, it would seem, like
the ``tale'' of the messenger, that this story is purely sentence,
telling only a moral tale and lacking any motion toward solace. It is
worth noting, however, that on quarter of the tale's length is spent on
the response of the skeptic/victim to the dreamer:

His felawe, that lay by his beddes syde, Gan to lauge, and scorned him
ful faste. `No dreem,' quod he, `may so myn herte agaste That I wol
lette for to do my thynges. I sette nat a straw by thy dremynges, For
swevenes been but vanytees and japes. Men dreme of thyng that nevere was
ne shal. But sith I see that thou wolt heere abyde, And thus
forswlewthne wilfully thy tyde, God woot, it reweth me; and have good
day!' (4276 - 4287)

The parallels here between the skeptic/victim and Pertelote are
striking. This whole speech echoes strongly Pertelote's earlier
arguments, with the specific claim that ``\ldots{} Swevenes been but
vanytees and japes'' pulling again, this time more closely, from the
same claim parodied in Chauntecleer's first tale, that, ``Nothyng, God
woot, but vanitee in sweven is.''

Chauntecleer's first tale, likely pulled from Cicero or Valerius Maximus
(cite), centers around two men traveling together on pilgrimage who are
forced to split up for the night, as there is no room for them anywhere.
One man is forced to spend the night in an ox's stall. The other dreams
three times during the night of the first man. In his first two dreams,
the other man begs for his assistance, claiming he is about to die. In
the third, after the dreamer has twice woken up and rejected his dreams
as ``nas but a vanitee,'' the first man informs him that he has already
died, and that the dreamer will find his corpse in a dung cart (4201).
In the morning, to his dismay, the dreamer finds all of this to be true;
the first man has been murdered, and his body is exactly where he
claimed it would be.

Each of Chauntecleer's tales has a dreamer, a skeptic, and a victim; the
first two tales split these roles between two characters, while in the
third, they all converge on the tale's only character. In this tale, the
dreamer is also the skeptic, the victim a prophetic figure who pleads
with him to take seriously the prophecy in his dreams. The moral of this
story seems relatively straightforward for Chauntecleer's purpose: one
should not consider such an urgent ream a ``vanitee,'' notably a direct
lifting of Pertelote's language at line 4112 that, ``Nothyng, God woot,
but vanitee in sweven is.'' At 4240, however, as Chauntecleer ends his
tale the moral impact of his narrative frays; rather than emphasizing
the wisdom of prophetic dreams, Chauntecleer gives praise to God for
outing the first man's murderer in overwrought, celebratory language
that departs significantly from the somber tone of the tale:

O blisful God, that art so just and trewe, Lo, how that thou biwreyest
mordre alway! Mordre wol out, that se we day by day. Mordre is so
wlatsom and abhomynable To God, that is so just and resonable, That he
ne wol nat suffre it heled be, Though it abyde a yeer, or two , or thre.
Mordre wol out, this my conclusion. And right anon, ministres of that
toun Han hent the cartere and so soore hym pyned, And eek the hostiler
so soor engyned, That they biknewe hire wikkednesse anon, And were
anhanged by the nekke-bon. Heere may men seen that dremes been to drede.
(4240-4253)

If the \emph{sentence} of this tale is straightforward, this ending
stanza shows the difficulty that Chauntecleer has communicating the
tale's \emph{solace}. His solution is problematic in that it introduces
an irrelevant notion that, in addition to taking away from the sentence
he otherwise successfully builds up, this solace essentially celebrates
not only the violent hanging of the murderer, but the murder itself, for
God's ability to uncover it. The necessity to insert solace into this
narrative allows for the redemption of the dreamer, despite his fault
leading to the death of his companion, ultimately minimizing the lesson
of the story, which is not to make this man's mistake. This stanza also
serves as a tangent which pulls the audience's attention away from the
dreamer's fault, the result of which is that the ending line, ``Heere
may men seen that dreams been to drede,'' seems out of place and
ultimately undeserved. The Canterbury Tales, by Geoffrey Chaucer, begins
with 30 pilgrims and a Host, Harry Bailey, who proposes to a game in
which each pilgrim tells four tales, two on the way down to Canterbury
and two on the way back. Bailey declares the winner, by his own
judgement, shall be whichever pilgrim tells tales, ``\ldots{}Of best
sentence and moost solaas'' (798). The resulting tales are an
interesting study not only in the many forms that sentence, primarily
understood as wisdom or moral instruction, and solace, or joy and
comfort, take in storytelling, but of the subjective production and
misprojection of sentence and solace across the stories told by distinct
individuals. At times parodic and at others celebratory, Chaucer
explores both the beauty and the danger of storytelling in such a
diverse company. No where is the dynamic exploration of storytelling
more clear than in the Nun's Priest's tale, which, in lampooning the
shortcomings of other pilgrims, simultaneously parodies the form of the
pilgrims' stories of the Canterbury Tales while playing Harry Bailey's
game perfectly, ultimately reinforcing the central movement of the
Canterbury Tales.

\end{document}
